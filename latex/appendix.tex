\chapter{网络性能指标}
\label{网络性能指标}

\&\#160; \&\#160; \&\#160; \&\#160;在前面几章中总是会涉及到相关网络中的术语,可能不利于读者理解,因此在最后加上附录,完善相关内容。

\section{速率}
\label{速率}

\&\#160; \&\#160; \&\#160; \&\#160;主要是指主机在数字通信上传送数据的速率,称为额定速率或者标称速率。
定义:单位时间(秒)传输信息量(比特)通常情况下用bit符号表示。
常用的单位:b\slash s, kb\slash s, Mb\slash s, Gb\slash s,他们之间的换算关系为:
$$1Gb/s = 10^6 Mb/s = 10^9 b/s$$

\section{吞吐率}
\label{吞吐率}

\&\#160; \&\#160; \&\#160; \&\#160;网络通信中发送端与接收端之间传送数据的速率。吞吐量受到网络的带宽或网络的额定速率的限制。

\begin{itemize}
\item 即时吞吐率:给定时刻的速率;

\item 平均吞吐率:一段时间的平均速率;

\item 瓶颈链路:端到端路劲上,带宽最小的链路。

\end{itemize}

\&\#160; \&\#160; \&\#160; \&\#160;吞吐率决定瓶颈链路的带宽,吞吐率为发送端的发送速率, 发送端发送数据的速率大于瓶颈链路,则吞吐率为瓶颈链路的带宽。

\section{延迟}
\label{延迟}

\&\#160; \&\#160; \&\#160; \&\#160;数据在传输过程中所消耗的时间即称为延迟,在分组交换网络中,延迟总共包含有四种:即节点处理延迟、排队延迟、传输延迟、传播延迟。

\begin{itemize}
\item {[1]} 节点处理延迟
节点(路由器)在处理数据时进行差错检测、确定链路输出等活动小号的时间,通常时间延迟很小,一般为毫秒级甚至更低。

\item {[2]} 排队延迟
需要传输的数据在节点中等待输出链路可用所花的时间,往往取决于节点(路由器)的拥塞程度。

\item {[3]} 传输延迟
节点将正在传输的分组数据发送到输出链路所用的时间,取决于L:分组长度(bit)和R:链路带宽,延迟d=L\slash R;

\item {[4]} 传播延迟
信号在信道中传播所用的时间,取决于信道长度d和电磁波信号在信道上的传播速度,延迟Pd=d\slash s

\item {[5]} 总时延
将上述的四个时延时间进行相加,既可以得到。

\end{itemize}

\section{丢包率}
\label{丢包率}

\&\#160; \&\#160; \&\#160; \&\#160;丢包数\slash 已发分组的总数,分组交换网络丢包的原因主要是节点的队列缓存容量有限,当分组到达时如果节点中的分组队列已满,则该分组会被丢弃,即称为丢包。

\section{带宽}
\label{带宽}

\begin{itemize}
\item 信号的带宽
信号的带宽是指信号所包含的各种不同频率成分所占据的频率范围。

\item 计算机网络的带宽
在单位时间内从网络中的某一点到另一点所能通过的最高数据率,即网络可通过的最高数据率,即每秒传输多少比特,单位是“比特每秒”,或b\slash s(bit\slash s)。

\end{itemize}

\section{时延带宽积}
\label{时延带宽积}

\&\#160; \&\#160; \&\#160; \&\#160;将两个网络性能的两个度量,传播时延和宽带相乘,就等到另外一个度量:传播时延带宽积(单位:bit;)。
$$时延带宽积=传播时延*带宽$$

\&\#160; \&\#160; \&\#160; \&\#160;下面是一个简单的例子:传播时延为20ms,带宽为10Mb\slash s,则时延带宽积 = 20 × 10 × 10\textsuperscript{3} \slash 1000 = 2 × 10\textsuperscript{5} bit。这就表示,若发送端连续发送数据,则在发送的第一个比特即将达到终点时,发送端就已经发送了20万个比特,而这20万个bit都在链路上向前移动。

\section{信道利用率}
\label{信道利用率}

\&\#160; \&\#160; \&\#160; \&\#160;信道利用率是指信道有百分之几的时间是被占用的比率,其计算公式如下:
$$S=发送时间T /(传播时延t+碰撞等待时间2n*t+发送时间)$$

\&\#160; \&\#160; \&\#160; \&\#160;上述提到的发送时间T是指:T=帧长数\slash 发送速率。
这里有一个简单例子加以说明,假设信道的长度为10km,往返传输时延为10ms,传输数据长度为2048bit,发送端的发送速率为1Mbps,在其他时延(碰撞等待时间)忽略的情况下,求信道利用率。

\begin{itemize}
\item 发送时间:2048\slash (1x1000x1000)=2.048ms

\item 信道利用率:2.048\slash (2.048+10)=17\%

\end{itemize}

\section{网络利用率}
\label{网络利用率}

\&\#160; \&\#160; \&\#160; \&\#160;网络利用率是指全网络的信道利用率的加权平均值。

\section{往返时间RTT}
\label{往返时间rtt}

\&\#160; \&\#160; \&\#160; \&\#160;表示从发送方发送数据开始,到发送方收到来自接收方的确认,总共经历的时间。
其与所发送的分组长度有关。发送很长的数据块的往返时间,应当比发送很短的数据块往返时间要多。
