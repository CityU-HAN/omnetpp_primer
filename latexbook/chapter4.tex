\chapter{OMNeT++编程接口}

\begin{summary}
在完成第五章后,考虑需要在之前加一章节关于<b>OMNeT++</b>类说明,在这个仿真软件中,主要使用的语言是<b>C++</b>,因此大多数数据类型是类或者结构,本章还是走其他技术书一样的老路线,注释这些数据类型,对类成员函数进行说明,可能与第五章有些重复的地方,但是其五章更多的偏向于实际应用,可能读者看过这里后,会发现<b>OMNeT++</b>接口是真好用。\\
\end{summary}

\section{循规蹈矩}


\section{类说明}



\subsection{cModule}
为了能更好的解释这个的库的使用,程序清单<b>4.1</b>为类<b>cModule</b>原型,<b>cModule</b>类在<b>OMNeT++</b>中表示一个节点的对象,这个节点可以是复合节点或者简单节点,通过这个类,程序员可以访问描述这个节点的<b>.ned</b>文件中设置的参数,或者是由<b>omnetpp.ini</b>传入的参数。简而言之,我们最后就是面向这些类进行网络设计。

\lstset{language=c}
\begin{lstlisting}
class SIM_API cModule : public cComponent //implies noncopyable
{
friend class cGate;
friend class cSimulation;
friend class cModuleType;
friend class cChannelType;

public:
/*
* 模块门的迭代器
* Usage:
* for (cModule::GateIterator it(module); !it.end(); ++it) {
*     cGate *gate = *it;
*     ...
* }
*/
class SIM_API GateIterator
{
...
};

/*
* 复合模块的子模块迭代器
* Usage:
* for (cModule::SubmoduleIterator it(module); !it.end(); ++it) {
*     cModule *submodule = *it;
*     ...
* }
*/
class SIM_API SubmoduleIterator
{
...
};

/*
* 模块信道迭代器
* Usage:
* for (cModule::ChannelIterator it(module); !it.end(); ++it) {
*     cChannel *channel = *it;
*     ...
* }
*/
class SIM_API ChannelIterator
{
...
};

public:
// internal: currently used by init
void setRecordEvents(bool e)  {setFlag(FL_RECORD_EVENTS,e);}
bool isRecordEvents() const  {return flags&FL_RECORD_EVENTS;}

public:
#ifdef USE_OMNETPP4x_FINGERPRINTS
// internal: returns OMNeT++ V4.x compatible module ID
int getVersion4ModuleId() const { return version4ModuleId; }
#endif

// internal: may only be called between simulations, when no modules exist
static void clearNamePools();

// internal utility function. Takes O(n) time as it iterates on the gates
int gateCount() const;

// internal utility function. Takes O(n) time as it iterates on the gates
cGate *gateByOrdinal(int k) const;

// internal: calls refreshDisplay() recursively
virtual void callRefreshDisplay() override;

// internal: return the canvas if exists, or nullptr if not (i.e. no create-on-demand)
cCanvas *getCanvasIfExists() {return canvas;}

// internal: return the 3D canvas if exists, or nullptr if not (i.e. no create-on-demand)
cOsgCanvas *getOsgCanvasIfExists() {return osgCanvas;}

public:

/** @name Redefined cObject member functions. */
//@{

/**
* Calls v->visit(this) for each contained object.
* See cObject for more details.
*/
virtual void forEachChild(cVisitor *v) override;

/**
* Sets object's name. Redefined to update the stored fullName string.
*/
virtual void setName(const char *s) override;

/**
* Returns the full name of the module, which is getName() plus the
* index in square brackets (e.g. "module[4]"). Redefined to add the
* index.
*/
virtual const char *getFullName() const override;

/**
* Returns the full path name of the module. Example: <tt>"net.node[12].gen"</tt>.
* The original getFullPath() was redefined in order to hide the global cSimulation
* instance from the path name.
*/
virtual std::string getFullPath() const override;

/**
* Overridden to add the module ID.
*/
virtual std::string str() const override;
//@}

/** @name Setting up the module. */
//@{

/**
* Adds a gate or gate vector to the module. Gate vectors are created with
* zero size. When the creation of a (non-vector) gate of type cGate::INOUT
* is requested, actually two gate objects will be created, "gatename$i"
* and "gatename$o". The specified gatename must not contain a "$i" or "$o"
* suffix itself.
*
* CAUTION: The return value is only valid when a non-vector INPUT or OUTPUT
* gate was requested. nullptr gets returned for INOUT gates and gate vectors.
*/
virtual cGate *addGate(const char *gatename, cGate::Type type, bool isvector=false);

/**
* Sets gate vector size. The specified gatename must not contain
* a "$i" or "$o" suffix: it is not possible to set different vector size
* for the "$i" or "$o" parts of an inout gate. Changing gate vector size
* is guaranteed NOT to change any gate IDs.
*/
virtual void setGateSize(const char *gatename, int size);

/*
* 下面的接口是关于模块自己的信息
*/
// 复合模块还是简单模块
virtual bool isSimple() const;

/**
* Redefined from cComponent to return KIND_MODULE.
*/
virtual ComponentKind getComponentKind() const override  {return KIND_MODULE;}

/**
* Returns true if this module is a placeholder module, i.e.
* represents a remote module in a parallel simulation run.
*/
virtual bool isPlaceholder() const  {return false;}

// 返回模块的父模块,对于系统模块,返回nullptr
virtual cModule *getParentModule() const override;

/**
* Convenience method: casts the return value of getComponentType() to cModuleType.
*/
cModuleType *getModuleType() const  {return (cModuleType *)getComponentType();}

// 返回模块属性,属性在运行时不能修改
virtual cProperties *getProperties() const override;

// 如何模块是使用向量的形式定义的,返回true
bool isVector() const  {return vectorSize>=0;}

// 返回模块在向量中的索引
int getIndex() const  {return vectorIndex;}

// 返回这个模块向量的大小,如何该模块不是使用向量的方式定义的,返回1
int getVectorSize() const  {return vectorSize<0 ? 1 : vectorSize;}

// 与getVectorSize()功能相似
_OPPDEPRECATED int size() const  {return getVectorSize();}


/*
* 子模块相关功能
*/

// 检测该模块是否有子模块
virtual bool hasSubmodules() const {return firstSubmodule!=nullptr;}

// 寻找子模块name,找到返回模块ID,否则返回-1
// 如何模块采用向量形式定义,那么需要指明index
virtual int findSubmodule(const char *name, int index=-1) const;

// 直接得到子模块name的指针,没有这个子模块返回nullptr
// 如何模块采用向量形式定义,那么需要指明index
virtual cModule *getSubmodule(const char *name, int index=-1) const;

/*
* 一个更强大的获取模块指针的接口,通过路径获取
*
* Examples:
*   "" means nullptr.
*   "." means this module;
*   "<root>" means the toplevel module;
*   ".sink" means the sink submodule of this module;
*   ".queue[2].srv" means the srv submodule of the queue[2] submodule;
*   "^.host2" or ".^.host2" means the host2 sibling module;
*   "src" or "<root>.src" means the src submodule of the toplevel module;
*   "Net.src" also means the src submodule of the toplevel module, provided
*   it is called Net.
*
*  @see cSimulation::getModuleByPath()
*/
virtual cModule *getModuleByPath(const char *path) const;

/*
* 门的相关操作
*/

/**
* Looks up a gate by its name and index. Gate names with the "$i" or "$o"
* suffix are also accepted. Throws an error if the gate does not exist.
* The presence of the index parameter decides whether a vector or a scalar
* gate will be looked for.
*/
virtual cGate *gate(const char *gatename, int index=-1);

/**
* Looks up a gate by its name and index. Gate names with the "$i" or "$o"
* suffix are also accepted. Throws an error if the gate does not exist.
* The presence of the index parameter decides whether a vector or a scalar
* gate will be looked for.
*/
const cGate *gate(const char *gatename, int index=-1) const {
return const_cast<cModule *>(this)->gate(gatename, index);
}


/**
* Returns the "$i" or "$o" part of an inout gate, depending on the type
* parameter. That is, gateHalf("port", cGate::OUTPUT, 3) would return
* gate "port$o[3]". Throws an error if the gate does not exist.
* The presence of the index parameter decides whether a vector or a scalar
* gate will be looked for.
*/
const cGate *gateHalf(const char *gatename, cGate::Type type, int index=-1) const {
return const_cast<cModule *>(this)->gateHalf(gatename, type, index);
}

// 检测是否有门
virtual bool hasGate(const char *gatename, int index=-1) const;

// 寻找门,如果没有返回-1,找到返回门ID
virtual int findGate(const char *gatename, int index=-1) const;

// 通过ID得到门地址,目前我还没有用到过
const cGate *gate(int id) const {return const_cast<cModule *>(this)->gate(id);}

// 删除一个门(很少用)
virtual void deleteGate(const char *gatename);


//返回模块门的名字,只是基本名字(不包括向量门的索引, "[]" or the "$i"/"$o")
virtual std::vector<const char *> getGateNames() const;

// 检测门(向量门)类型,可以标明"$i","$o"
virtual cGate::Type gateType(const char *gatename) const;

// 检测是否是向量门,可以标明"$i","$o"
virtual bool isGateVector(const char *gatename) const;

// 得到门的大小,可以指明"$i","$o"
virtual int gateSize(const char *gatename) const;

// 对于向量门,返回gate0的ID号
// 对于标量ID,返回ID
// 一个公式:ID = gateBaseId + index
// 如果没有该门,抛出一个错误
virtual int gateBaseId(const char *gatename) const;

/**
* For compound modules, it checks if all gates are connected inside
* the module (it returns <tt>true</tt> if they are OK); for simple
* modules, it returns <tt>true</tt>. This function is called during
* network setup.
*/
virtual bool checkInternalConnections() const;

/**
* This method is invoked as part of a send() call in another module.
* It is called when the message arrives at a gates in this module which
* is not further connected, that is, the gate's getNextGate() method
* returns nullptr. The default, cModule implementation reports an error
* ("message arrived at a compound module"), and the cSimpleModule
* implementation inserts the message into the FES after some processing.
*/
virtual void arrived(cMessage *msg, cGate *ongate, simtime_t t);
//@}

/*
* 公用的
*/
// 在父模块中寻找某个参数,没找到抛出cRuntimeError
virtual cPar& getAncestorPar(const char *parname);

/**
* Returns the default canvas for this module, creating it if it hasn't
* existed before.
*/
virtual cCanvas *getCanvas() const;

/**
* Returns the default 3D (OpenSceneGraph) canvas for this module, creating
* it if it hasn't existed before.
*/
virtual cOsgCanvas *getOsgCanvas() const;

// 设置是否在此模块的图形检查器上请求内置动画。
virtual void setBuiltinAnimationsAllowed(bool enabled) {setFlag(FL_BUILTIN_ANIMATIONS, enabled);}

/**
* Returns true if built-in animations are requested on this module's
* graphical inspector, and false otherwise.
*/
virtual bool getBuiltinAnimationsAllowed() const {return flags & FL_BUILTIN_ANIMATIONS;}
//@}

/** @name Public methods for invoking initialize()/finish(), redefined from cComponent.
* initialize(), numInitStages(), and finish() are themselves also declared in
* cComponent, and can be redefined in simple modules by the user to perform
* initialization and finalization (result recording, etc) tasks.
*/
//@{
/**
* Interface for calling initialize() from outside.
*/
virtual void callInitialize() override;

/**
* Interface for calling initialize() from outside. It does a single stage
* of initialization, and returns <tt>true</tt> if more stages are required.
*/
virtual bool callInitialize(int stage) override;

/**
* Interface for calling finish() from outside.
*/
virtual void callFinish() override;


/*
* 动态模块创建
*/

/**
* Creates a starting message for modules that need it (and recursively
* for its submodules).
*/
virtual void scheduleStart(simtime_t t);

// 删除自己
virtual void deleteModule();

// 移动该模块到另一个父模块下,一般用于移动场景。规则较复杂,可到原头文件查看使用说明
virtual void changeParentTo(cModule *mod);
};



\end{lstlisting}

\subsection{cPar}


\subsection{cGate}


\subsection{cTopology}


\subsection{cExpression}



