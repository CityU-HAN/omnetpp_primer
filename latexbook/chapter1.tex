%---------------------------------------------------------------------------
\chapter{OMNeT++}

\begin{summary}
OMNeT++是一个网络仿真工具,支持以太网、无线网等协议仿真,同时提供友好的仿真界面以及3D显示。
\\
\end{summary}

\section{OMNeT++简介}

OMNeT++,一个基于eclipse开发套件的开源网络仿真工具,目前主要在高校实验室进行一些网络仿真测试,对一些算法进行对比,它可以供使用者进行完成以下开发:
\begin{itemize}
\item C/C++开发;
\item 网络仿真程序设计。
\end{itemize}
毫无疑问,基于<b>eclipse</b>的开发工具肯定能支持普通的<b>C/C++</b>工程。
另外,在<b>OMNeT++</b>上网络仿真设计领域的优势在于,它是一个开源的项目,对大量的网络模型都提供代码支持。但是问题在于国内的确没有什么社区支持,出现问题只能自己解决,其实对于开源的项目大多存在这种问题,往往开源的项目,使用起来难度较大,开源项目往往比那些商业的软件开发难度较大,支持也较少,开源可不代表简单。</br>
<b>OMNeT++</b>对初学者能力要求高,它假定使用者对编程有一定了解的,对eclipse开发环境也是特别熟悉的,另外这是一个网络仿真的软件,需要你对计算机网络有足够的认识,它提供了大量现有各种网络的仿真例子,如果你对网络认识足够强,那么这个软件你用起来会感到特别顺手。
\\
\section{我的初衷}

这是一个计划,计划在2018年写一个OMNeT++编程指导手册。

\begin{definition}
  由于我在学习使用\bm{$OMNeT++$}的过程中,遇到很多问题,虽然在安装的过程中,没有遇到什么阻碍,如果只是简单的想仿真自带的例子,估计就只需要修改仿真程序配置文件,在加上能分析仿真结果就行,但是想要完全自己写一个仿真程序,这些是完全不够的,这方面可以上YouTobe上搜索,一堆OMNeT++仿真的程序,基本都是自己开发的,而在国内,各论文上的OMNeT++仿真应该是自己写的,但是基本不会提供源代码。\newline
  我在后面的网络设计过程中,遇到很多疑问,每遇到一个问题都是花了两三个小时才解决,其中一些问题,也就是设置问题。特此总结一下我在OMNeT++里边踩的坑。
\end{definition}


\section{目录}

本手册与现有的那两本书风格不同,我希望读者通过此手册可以快速的上手OMNeT++,快速的掌握OMNeT++提供的各种接口,目前包括以下内容:

\begin{itemize}
	\item OMNeT++的安装
	\item INET库的安装 INET库的基本使用
	\item OMNeT++个性化设置
	\item OMNeT++工程设计技巧
	\item cModule | cPar | cGate | cTopology相关类使用
	\item 仿真结果分析
	\item 仿真错误记录
\end{itemize}

%---------------------------------------------------------------------------


